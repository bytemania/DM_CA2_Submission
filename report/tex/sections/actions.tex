\section{Actions}

Our model has pinpointed the primary factors contributing to severe 
COVID-19 cases. It appears that severe scenarios are more common among 
older patients, particularly those with comorbidities such as diabetes 
or pneumonia. Based on our findings, it is crucial to focus protective 
measures on the elderly, as they are the most vulnerable group. There
are existing measures that can be enhanced to safeguard this 
population in the event of a pandemic:
\begin{description}
    \item [Vaccination \& Education] Ensure vaccines are easily
    accessible to the vulnerable groups. Educate the population on the
    benefits of vaccination;
    \item[Rapid Testing] Deploy mobile testing units to areas with 
    outbreaks;
    \item[Contact Tracing] Leverage digital tools and apps to improve
    the speed of contact tracing;
    \item[Distancing] Using masks in crowded spaces, avoiding public
    spaces and limit the contact for the older population;
    \item[Hospital Infrastructure] Augment resources in hospitals: ICU
    beds, ventilators and trained healthcare personal.
\end{description}

These actions when implemented effectively can improve the response
to COVID-19, helping to reduce the transmission rates, support the 
healthcare systems, and ultimately save lives.

Triage for COVID-19 involves prioritizing medical care and resources 
based on the severity of patients' symptoms and their potential 
risk for serious complications. Effective triage is crucial during a 
pandemic to ensure that healthcare systems can manage the influx of 
patients without becoming overwhelmed. 

To ensure effective triage, we should continue to refine and expand our
model, enabling it to provide rapid feedback across various scenarios
and assist with resource allocation. Quick response and feedback are 
essential in every situation, particularly during a pandemic. Relying 
solely on human analysis to manage all cases is impractical and 
unsustainable. Therefore, developing classification models is crucial 
to aid in the decision-making process, helping determine which 
patients require immediate medical attention in urgent care settings.
