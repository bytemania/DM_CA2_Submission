\section{Reflections}

Our study highlights the complex relationship between COVID-19, 
infectious diseases, and chronic health conditions such as diabetes and
pneumonia.

The link between diabetes and COVID-19 is particularly noteworthy, as 
individuals with diabetes face a heightened risk of severe outcomes
when infected with the virus. Similarly, pneumonia is recognized as 
one of the most severe complications associated with COVID-19, 
often leading to critical cases.

Age also plays a crucial role and is a strong predictor of the need for
hospitalization. Patients aged over 50 are significantly more likely to 
require hospital treatment. This increased vulnerability is partly due 
to the diminished immune response seen in older individuals, coupled 
with a higher incidence of comorbid conditions that exacerbate the 
severity of COVID-19 outcomes.

We were dissatisfied with the model's inability to accurately predict 
hospitalization needs, achieving only a 61.1\% accuracy rate. 
Our primary concern is the 6.2\% of patients who require treatment 
but were mistakenly not recommended for hospitalization by our 
model. This issue largely stems from the imbalanced nature of 
our dataset. 

To reduce errors in patient triage, we employed a stacking technique to 
enhance prediction accuracy by integrating multiple algorithms. We organized
the features by category as outlined in Section~\ref{subsec:stack}. This
approach successfully decreased the error margin from 6.2\% to 4\%, marking
a substantial improvement.

Improving the dataset is another effective strategy for enhancing the 
accuracy of predictions in patient triage.
To enhance the quality of our data and potentially 
improve the model's performance, we propose the following steps:

\begin{description}
    \item[Increase the dataset size] Given the complexity of the 
    scenario, a dataset of 1 million patients may not be sufficient.
    Increasing the dataset size is essential, particularly to better 
    represent minority scenarios such as patients requiring 
    hospitalization. This expansion will enable the model to learn
    more effectively from these critical cases;
    \item[International Expansion] Extending the study to include data 
    from countries beyond Mexico could enhance the robustness of the 
    model. Incorporating international datasets would provide a more 
    diverse range of patient profiles and health dynamics, contributing
    to improved model accuracy;
    \item[Balance the Dataset] There is a need to adjust the dataset to
    include more cases that involve hospitalization. A more balanced 
    dataset will help mitigate the current model's bias toward the 
    majority class and improve its predictive accuracy for critical care 
    scenarios;
    \item[Enrich the Feature Set] Adding more variables could 
    significantly enhance the model's understanding of the patients. 
    Features such as socioeconomic status, living conditions, and 
    vaccination status are vital as they can influence the likelihood of
    hospitalization and the overall course of treatment. These
    additional features will provide a more comprehensive view of each 
    patient, leading to better-informed predictions.
\end{description}

Implementing these strategies, we can enhance our model’s ability to 
accurately predict which patients require hospitalization, thereby 
improving patient outcomes and optimizing resource allocation.



