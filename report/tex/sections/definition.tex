\section{Definition of the problem}

The COVID-19 pandemic placed a strain in the health services in most countries. The disease is very 
infectious, resulting in sudden spikes in case numbers which quickly overwhelmed hospitals.
The doctors, nurses and other front line health professionals were critical for responding to the 
pandemic, however their work placed them in extreme risk of infection. The health services ability 
to respond and ensure the best possible outcomes to patients depended on the health of front line
staff.

Every effort was made with PPE equipment and operational procedures to protect staff. A critical
aspect of this was to reduce contact between patients and staff and ensure that as many patients
could be treated at home as possible.

The Mexican government have sponsored a data mining study to examine how simple health data could
be used to automate the triage of COVID-19 patients. They hope this will improve their response to
new waves of COVID-19 or similar respiratory illness. The goal is to improve their triage process
and safely reduce the number of patients that require contact with a health professional.

Initially patients are triaged outside the hospital to limit the chance of cross infection. Normally 
patients are triaged by a health professional as soon as they present, this person decides if they
need to be admitted to hospital for treatment. However before they are triaged
patients will be asked to complete a simple questionnaire. This data together with a COVID-19 test 
result will enable a non-health professional to efficient route them as follows:
 \begin{itemize}
     \item Low risk patients will be sent home without further contact with front line staff;
     \item High risk patients will be streamlined for hospital care with the least possible interaction;
     \item If the risk assessment is undetermined patients will triaged by a health professional who will
     make the final decision on admission.
 \end{itemize}

While the goal is to minimise the number of patients requiring triage it is very important to do so
safely and with the utmost care for the patients. Therefore false negatives where patients are
sent home in error must be minimised. Likewise false positives of where patients are hospitalised in
error must be minimised to preserve capacity. If there is any doubt the patient must be triaged by
a health professional.

The government has provided a dataset~\parencite[]{2023:NizMei} with details and test results for 
1,048,575 previous COVID-19 patients. This is used to develop a predictive model to identify 
patients who require admission to hospital, no admission and those with require further triage.

A well trained model can aid medical staff in making quick data-driven decisions helping:
 \begin{itemize}
     \item Medical resources prioritization;
     \item Enhance patient care;
     \item Increase the survival rate.
 \end{itemize}
We can also correlate some key conditions with some patients conditions such as:
 \begin{itemize}
     \item Age;
     \item Obesity;
     \item Diabetes;
     \item Hypertension.
 \end{itemize}
 